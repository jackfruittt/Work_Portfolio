\newpage
\section{S100: Demonstration robot for the UTS Robotics Institute} 
\label{sec:s100}
The S100 demonstration robot [~\ref{fig:s100}] is an incomplete project before my departure from the UTS Robotics Institute. It was a robotic platform smaller than RIMA2 intented to do similar tasks to RIMA1 and RIMA2. I designed The robot
was desiged with the following featrures:

\begin{itemize}
    \item Single Teensy MCU for control
    \item Communicated with base station using an MQTT broker over ethernet to replicate similar communcation functionality to RIMA1 and RIMA2
    \item GUI with intuitive controls and a 3D real-time imu visualisation all devolped using ReactJS [~\ref{fig:s100gui}]
    \item The ability to be used fully without ROS in Windows and Linux to make it more user friendly for non-technical users; the GUI can be built in production as a standalone application using ElectronJS
    \item The GUI was also linux compatible and a ROS bridge could be used to also control the Robot in ROS Noetic amongst other things, making the robot versatile for both general and research use. Theoretically also functional in ROS2 with adjusted logic.
    \item Tof and 2D LiDaR was going to be implemented however my contract ended before this could be completed however, I do have calibration/interpolation scripts.
\end{itemize}

\textbf{Codebase: } The repository for the S100 project can be found \href{https://github.com/jackfruittt/S100_Interface}{here} and the \href{https://github.com/jackfruittt/tmf8801_Calibration}{tof calibration}
and \href{https://github.com/jackfruittt/Hokuyo_Calibration}{LiDaR calibration}

\newpage
\imginin{s100/s100_robot.png}{S100 Robot}{s100}{0.4}
\imginin{s100/webui.png}{S100 GUI}{s100gui}{0.8}