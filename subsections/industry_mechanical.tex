% Tether Protetiion Section
\newpage
\subsection{Mechanical Work}

%\textbf{Utilities for RIMA1 and RIMA2 Field Trials:} 
\subsubsection{Utilities for RIMA1 and RIMA2 Field Trials:}
I designed a number of utilities to assist with field trial deployments to prevent damage to equipment and site. Using Solidworks I designed
tether protection for flangless and flanged pipes [~\ref{fig:tethproc}] which were later manufactured to be used in deployments. The tether protection units are designed with the following intent:

\begin{itemize}
    \item Ease of use for Sydney Water personnel
    \item Prevent damage to the tether and pipe wall at entrance where edge is sharp
    \item Adjustability for different pipe diameters (250mm to 750mm)
    \item Stainless steel for Chemical and Weather resitance and durability
    \item Simple lock-pin mechanism to ensure ease of use when wearing gloves
    \item Teflon block with internal radius matched to the bend radius of tether cable connected from base station to the robot
    \item Simple, low cost design
\end{itemize}

\imgpairin{utility/flangless_tether_protection.JPG}{Flangless Tether Protection}{utility/tether_protection_flanged.JPG}{Flanged tether protection}{Tether Protection}{tethproc}{90}{270}

% Deployment Cradle Section
\newpage
\subsubsection{RIMA2 Deployment Cradle:}
The next utility I designed was a deployment cradle for RIMA2 [~\ref{fig:jdm}]. Often, due to safety constraints only one person is allowed into the sewer pit. This does not favour the handling of RIMA2 which is a 3-body 
system best carried and inserted into pipes safely by two people. There was an incident during a field trial where RIMA2 was dropped, I was assigned with designing the cradle. The advantage 
of the cradle is that it can safely lower and deploy RIMA2. The final deisgn features:

\begin{itemize}
    \item Ability to be lowered using block-and-chain or cranes (varies with resource availability on site)
    \item Lightweight to ensure weight requirements are met as per agreement with Sydney Water
    \item Simple, intuitive locking mechanisms to ensure ease of use for Sydney Water personnel
    \item Custom flange design to adapt to various flanged pipe diameters (250mm to 450mm) if additional support is required
\end{itemize}

\imgpairin{utility/jdm.jpg}{RIMA2 Cradle in lab}{utility/jdm_in_use.jpg}{Cradle in use on field trial}{RIMA2 Cradle}{jdm}{0}{270}

% Rollover-Assist RIMA1
\newpage
%\textbf{RIMA1 Rollover Assist:} 
\subsubsection{RIMA1 Rollover Assist:}
I designed a set of rollover assist devices for RIMA1 after an incident in one of the first trial runs with Sydney Water during the handover period
a software bug caused the robot to drive autonomously up the pipe wall causing in to invert. The aim of the rollover assist is to help with manually pulling the robot out the pipe when it inverts, as 
using machinery to dig up the robot can be costly, time-consuming and potentially impossible depending on location.The inital inspiration [~\ref{fig:rp}] actually came from when I pulled apart my skateboard and attached it to RIMA1.
The final design [~\ref{fig:fdra}] after a few iterations is quite different from this and is simply a modification of the Sensor Pad on the sensor head with a custom bracket that could be fitted to RIMA1 with simple modification to the lifting handles.  

\imginin{rima1/rollover_prototype.jpg}{Rollover Prototype}{rp}{0.29}
\imgpairin{rima1/rollover_rear.jpg}{Rollover Assist Rear View}{rima1/rollover_side.jpg}{Rollover Side View}{Final Design Rollover Assist}{fdra}{270}{0}
